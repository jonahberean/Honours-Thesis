%% The following is a directive for TeXShop to indicate the main file
%%!TEX root = diss.tex

%%%%%%%%%%%%%%%%%%%%%%%%%%%%%%%%%%%%%%%%
\chapter{Results and Discussion}
\label{ch:ResultsandDiscussion}
%%%%%%%%%%%%%%%%%%%%%%%%%%%%%%%%%%%%%%%%

The results of the experimental measurements are presented in this chapter. In Section \ref{sec: Determining Internal Conversion Coefficients} it was shown how previously unmeasured internal conversion coefficients may be determined. These new measurements are detailed in Section \ref{sec: Measured E0 Transition Strengths}. For the $2^+_2 \rightarrow 2^+_1$ transition in $^{110}\mathrm{Pd}$, a newly measured internal conversion coefficient was combined with other experimental data to calculate the $E0$ transition strength. In Section \ref{sec: Measured E0 Transition Strengths} this result is presented within the context of similar measurements performed in other isotopes of Pd. 

%%%%%%%%%%%%%%%%%%%%%%%%%%%%%%%%%%%%%%%%
\section{Measured Internal Conversion Coefficients}
\label{sec: Measured Internal Conversion Coefficients}
%%%%%%%%%%%%%%%%%%%%%%%%%%%%%%%%%%%%%%%%

The measured internal conversion coefficients (for the $K$ shell) are presented in Table \ref{table:measured_ICC} in conjunction with theoretical values from BrIcc \cite{KIBEDI2008202}. Comparison with the BrIcc values shows that certain multipolarities are less probable than others for the measured transitions. However, the measurements do not provide a definitive argument as to the exact multipolarities of any of the measured transitions.

\begin{table}
  \begin{center}
    \begin{tabular}{|l|l l|l|l l l l|} 
     \hline
     & Transition & $E_\gamma \ [\mathrm{keV}]$ & $\alpha_\mathrm{exp}(K)$ & \multicolumn{4}{c|}{BrIcc $\alpha(K)$} \\
     & \multicolumn{2}{c|}{} &  										& $E1$		& $M1$		& $E2$		& $M2$		\\  
     \hline 
     $^{109}\mathrm{Pd}$ & $1/2, 3/2 \rightarrow 5/2^-$ 			& 530.202 	& 7(2)	& 1.44(2)	& 4.24(6)	& 4.38(7)	& 13.3(2)	\\ 
     \hline 
     $^{110}\mathrm{Pd}$ & $(3^+) \rightarrow 2^+_2$ 				& 398.6   	& 14(3) & 2.90(4)	& 8.5(1) 	& 10.2(2) 	& 31.0(5) 	\\  
                         & $(6^+) \rightarrow 4^+_2$ 				& 588.8   	& 5(2) 	& 1.13(2) 	& 3.31(5) 	& 3.27(5) 	& 9.9(1)	\\ 
                         & $(4^+) \rightarrow (3^+)$ 				& 687.7   	& 4(2) 	& 0.81(1)	& 2.30(4) 	& 2.17(3)	& 6.41(9) 	\\    
     \hline 
     $^{111}\mathrm{Ag}$ & $11/2^{(+)} \rightarrow 9/2^+_1$ 		& 575.1 	& 5(3)& 1.27(2)	& 3.81(6)	& 3.68(6)	& 11.5(2)	\\  
                         & $11/2^+, 13/2_1^+ \rightarrow 9/2_1^+$ 	& 694.19    & 5(1)& 0.84(1) 	& 2.45(4) 	& 2.24(4) 	& 6.8(1)	\\  
     \hline
    \end{tabular}
  \end{center}
  \caption[Measured internal conversion coefficients for the $K$ electron shell, $\alpha_\mathrm{exp}(K)$, compared to the theoretically calculated values from BrIcc \cite{KIBEDI2008202}.]{Measured internal conversion coefficients for the $K$ electron shell, $\alpha_\mathrm{exp}(K)$, compared to the theoretically calculated values from BrIcc \cite{KIBEDI2008202}. All internal conversion coefficient values are given in milliunits. Calculated values from BrIcc are denoted by the multipolarity assumed for the transition. $E_\gamma$ is the transition energy. The enclosure of a nuclear level's spin and/or parity assignment, such as $(3^+)$, indicates a tentative assignment.}
  \label{table:measured_ICC}
\end{table}

Two measurements were made of $K$ shell internal conversion coefficients for mixed transitions. These two results are presented in Table \ref{table:measured_mixed_ICC}.

\begin{table}
  \begin{center}
    \begin{tabular}{|l|l l|l|l|l l l|} 
     \hline
     & Transition & $E_\gamma \ [\mathrm{keV}]$ &$\delta(E2/M1)$& $\alpha_\mathrm{exp}(K)$ & \multicolumn{3}{c|}{BrIcc $\alpha(K)$} \\
     &  \multicolumn{2}{c|}{}	& 		& 			& $E2$		& $M1$		& $(E2/M1)$	\\  
     \hline 
     $^{110}\mathrm{Pd}$	& $2_2^+ \rightarrow 2_1^+$ & 439.76 & $-4.6^{+1.9}_{-1.2}$	& 7.4(5)	& 7.6(1) 	& 6.6(1)	& 7.5(1)	\\  
                         	& $4_2^+ \rightarrow 4_1^+$ & 477.8  & Unmeasured			& 7(3)		& 5.9(1) 	& 5.4(1)	& 5.7(3)	\\   
     \hline
    \end{tabular}
  \end{center}
  \caption[Measured internal conversion coefficients for known mixed transitions, $\alpha_K$ (for the K electron shell), compared to the theoretically calculated values from BrIcc \cite{KIBEDI2008202}.]{Measured internal conversion coefficients for known mixed transitions, $\alpha_K$ (for the K electron shell), compared to the theoretically calculated values from BrIcc \cite{KIBEDI2008202}. All internal conversion coefficient values are given in milliunits. For the $4_2^+ \rightarrow 4_1^+$ transition in $^{110}\mathrm{Pd}$ the BrIcc default value of 1.00 for $E2/M1$ mixing is used. $E_\gamma$ is the transition energy.}
  \label{table:measured_mixed_ICC}
\end{table}

The measured internal conversion coefficient for the $2_2^+ \rightarrow 2_1^+$ transition is consistent with the prediction from BrIcc. There is little that can be said about the measured internal conversion coefficient for the $4_2^+ \rightarrow 4_1^+$ transition, as no mixing ratio has been measured. 

%%%%%%%%%%%%%%%%%%%%%%%%%%%%%%%%%%%%%%%%
\section{Measured $E0$ Transition Strengths}
\label{sec: Measured E0 Transition Strengths}
%%%%%%%%%%%%%%%%%%%%%%%%%%%%%%%%%%%%%%%%

Using the experimental values of the internal conversion coefficients measured in this work, in combination with other experimental and theoretical values from literature \cite{KIBEDI2008202,ENSDF110Pd}, the $E0$ transitions strengths were evaluated for the $2_2^+ \rightarrow 2_1^+$ transition in $^{110}\mathrm{Pd}$. In Table \ref{table: Comparing E0 Strengths of Pd Nuclei}, the measurement resulting from the Monte Carlo analysis and Neyman construction is given in comparison to other $\rho^2(E0)$ measurements in the Pd nuclei. For transitions where there are two existing measurements of the $(E2/M1)$ mixing ratios, both values and their corresponding $\rho^2(E0)$ measurements are given. The Monte Carlo error analysis was carried out with $10^8$ simulation events. The bin size of the resulting $\rho^2(E0)$ distribution was 0.05 milliunits. In the construction of the Neyman plots (using the Feldman-Cousins ordering principle), each vertical row of the `true' mean value was calculated every 0.001 milliunits.

Additionally, an upper limit has been given for the $\rho^2(E0)$ value in the $4^+_2 \rightarrow 4^+_1$ transition of $^{110}$Pd. A maximal value for the mixing ratio has been assumed for this calculation, implying a transition that has a negligible $M1$ component. The calculation of this limit followed the same Monte Carlo and Neyman methodology as detailed for the $2^+_2 \rightarrow 2^+_1$ transition. 

\begin{sidewaystable}
\begin{center}
\begin{tabular}{|l|l l l l|l l l|} 
\hline
& Transition & $E_\gamma \ [\mathrm{keV}]$ & $\mathrm{T}_{1/2} \  [\mathrm{ps}]$ & $\delta(E2/M1)$ & $\alpha_\mathrm{exp}(K) \times 10^3 $ & $q^2_K(E0/E2)$ & $\rho^2(E0) \times 10^3$\\
\hline
$^{102}\mathrm{Pd}$&$0_2^+\rightarrow 0_1^+$&1592.6 &$14.5(4)\times 10^3$	&&&$>2$        & $4.0(15)$\\
                   &$0_3^+\rightarrow 0_1^+$&1658.1 &$0.87(22)$  			&&&$<0.0014$ & $<0.2$\\
\hline
$^{104}\mathrm{Pd}$&$0_2^+\rightarrow 0_1^+$&1333.6 &$5.2(5)$    			&&&$0.12(6)$	& $6(4)$\\ 
\hline 


$^{106}\mathrm{Pd}$&$0_2^+\rightarrow 0_1^+$&1133.7	&$5.813)$ &&&& $14(4)$\\
                   &$2_3^+\rightarrow 2_1^+$&1562.2	&$1.3(2)$  & $+0.24(1)$	&$1.12(15)$	&& $34(22)$\\   
                   &						 &	   	&									&&$1.56(24)$&& $97(39)$\\                   
                   &$0_3^+\rightarrow 0_1^+$&1194.5	&$2.8(5)$    			&& 					 		&& $<3$\\                  
                   &$2_4^+\rightarrow 2_1^+$&1397.4	&$1.1^{+10}_{-2}$					&$1.32^{+11}_{-25}$			&$0.68(9)$ 	&& $21^{+10}_{-21}$\\                   
                   &						&	   	&									&$0.17^{+4}_{-1}$  			&&& $18^{+10}_{-18}$\\                   
                   &$4_2^+\rightarrow 4_1^+$&703.11	&$0.50^{+24}_{-2}$    				&$0.66^{+48}_{-33}$			&$1.46(26)$	&& $96^{+43}_{-61}$\\                   
\hline
$^{108}\mathrm{Pd}$&$0_2^+\rightarrow 0_1^+$&1052.8	&$4.0(4)$    			&&&$<0.02$              	& $<3$\\
\hline 
$^{110}\mathrm{Pd}$&$0_2^+\rightarrow 0_1^+$&946.7 	&$7.9(7)$    			&&&$0.034(6)$ 	& $4.0(8)$  	\\
                   &$2_2^+\rightarrow 2_1^+$&439.76	&$17.7(8)$   			& $-4.6^{+1.9}_{-1.2}$  	& $7(1)^\dagger$  	& --- & $<9^\dagger$\\
                   &$4_2^+\rightarrow 4_1^+$&477.8	&$5.1(6)$    			& Unmeasured            	& $7(3)^\dagger$ & --- & $<129^\dagger$\\
\hline
\end{tabular}
\end{center}
\caption[A comparison of $E0$ transition measurements across the Pd nuclei. New measurements resulting from this work are highlighted.]{A comparison of $E0$ transition measurements across the Pd nuclei. In the nucleus $^{106}$Pd, multiple $\rho^2(E0)$ measurements have been made, and listed here, for cases where there are differing published values of $\delta(E2/M1)$ or $\alpha_\mathrm{exp}(K)$. The half-life, $T_{1/2}$, values given are for the transition parent state. The determination of a limit on $\rho^2(E0)$ for the $2_2^+ \rightarrow 2_1^+$ transition in $^{110}\mathrm{Pd}$ is consistent with the expectation of little to no shape mixing. The magnitude of the limit is consistent with that of other Pd isotopes \cite{Peters2016,Kibedi2005}. The $\rho^2(E0)$ limit determined on the $4^+_2\rightarrow4^+_1$ assumes a maximal value for the unmeasured $\delta(E2/M1)$, hence a negligible $M1$ component. $^\dagger$This work.} 
\label{table: Comparing E0 Strengths of Pd Nuclei}
\end{sidewaystable}

The measured $\rho^2(E0)$ limit for the $2_2^+ \rightarrow 2_1^+$ transition in $^{110}$Pd is consistent with the expectation of there being negligible mixing between these two intrinsic excitation bands. In comparing to the other Pd nuclei for which published measurements exist, the small magnitude of the limit appears to be consistent with the $\rho^2(E0)$ values. For the $4^+_2 \rightarrow 2^+_1$ transition in $^{110}$Pd, the lack of a measured mixing ratio yields a relatively inexact upper bound value of $<129$. 

\endinput

Any text after an \endinput is ignored.
You could put scraps here or things in progress.

% !!! A version of the final table where measurements are more rigorously cited. Simplifying things as much as possible for the draft.

\begin{sidewaystable}
\begin{center}
\begin{tabular}{|l|l l l l|l l l|} 
\hline
& Transition & $E_\gamma \ [\mathrm{keV}]$ & $\mathrm{T}_{1/2} [ps]$ & $\delta(E2/M1)$ & $\alpha_K \times 10^3 $ & $q^2_K(E0/E2)$ & $\rho^2(E0) \times 10^3$\\
\hline
$^{102}\mathrm{Pd}$&$0_2^+\rightarrow 0_1^+$&1592.6 &$14.54})\times 10^3$	&&&$>2$\textsuperscript{a}        & $4.015})$\textsuperscript{a}\\
                   &$0_3^+\rightarrow 0_1^+$&1658.1 &$0.8722})$  			&&&$<0.0014$\textsuperscript{a} & $<0.2$\textsuperscript{a}\\
\hline
$^{104}\mathrm{Pd}$&$0_2^+\rightarrow 0_1^+$&1333.6 &$5.25})$    			&&&$0.126})$\textsuperscript{a}	& $64})$\textsuperscript{a}\\ 
\hline 


$^{106}\mathrm{Pd}$&$0_2^+\rightarrow 0_1^+$&1133.7	&$5.813})$ &&&& $144})$\textsuperscript{a}\\
                   &$2_3^+\rightarrow 2_1^+$&1562.2	&$1.32})$  & $+0.241})$\textsuperscript{b}	&$1.1215})$\textsuperscript{c}	&& $3422})$\textsuperscript{d}\\   
                   &						 &	   	&									&&$1.5624})$\textsuperscript{e}&& $9739})$\textsuperscript{d}\\                   
                   &$0_3^+\rightarrow 0_1^+$&1194.5	&$2.85})$    			&& 					 		&& $<3$\textsuperscript{d}\\                  
                   &$2_4^+\rightarrow 2_1^+$&1397.4	&$1.1^{+10}_{-2}$					&$1.32^{+11}_{-25}$\textsuperscript{d}			&$0.689})$\textsuperscript{c} 	&& $21^{+10}_{-21}$\textsuperscript{d}\\                   
                   &						&	   	&									&$0.17^{+4}_{-1}$\textsuperscript{d}  			&&& $18^{+10}_{-18}$\textsuperscript{d}\\                   
                   &$4_2^+\rightarrow 4_1^+$&703.11	&$0.50^{+24}_{-2}$    				&$0.66^{+48}_{-33}$\textsuperscript{d}			&$1.4626})$\textsuperscript{c}	&& $96^{+43}_{-61}$\textsuperscript{d}\\                   
\hline
$^{108}\mathrm{Pd}$&$0_2^+\rightarrow 0_1^+$&1052.8	&$4.04})$    			&&&$<0.02$\textsuperscript{a}              	& $<3$ \textsuperscript{a}\\
\hline 
$^{110}\mathrm{Pd}$&$0_2^+\rightarrow 0_1^+$&946.7 	&$7.97})$    			&&&$0.034 (\textsl{6})$\textsuperscript{a} 	& $4.08})$\textsuperscript{a}  	\\
                   &$2_2^+\rightarrow 2_1^+$&439.76	&$17.78})$   			& $-4.6^{+1.9}_{-1.2}$\textsuperscript{e}  	&$71})$\textsuperscript{e}    	&& $<9$\textsuperscript{e}\\
                   &$4_2^+\rightarrow 4_1^+$&477.8	&$5.16})$    			& Unmeasured            	&&& $<36$\textsuperscript{e}\\
\hline
\end{tabular}
\end{center}
\caption{A comparison of $E0$ transition measurements across the Pd nuclei. The half-life, $T_{1/2}$, values given are for the transition parent state. } \textsuperscript{*}From ref. \cite{Kibe}
\label{table: Comparing E0 Strengths of Pd Nuclei}
\end{sidewaystable}
