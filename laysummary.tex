%% The following is a directive for TeXShop to indicate the main file
%%!TEX root = diss.tex

%% https://www.grad.ubc.ca/current-students/dissertation-thesis-preparation/preliminary-pages
%% 
%% LAY SUMMARY Effective May 2017, all theses and dissertations must
%% include a lay summary.  The lay or public summary explains the key
%% goals and contributions of the research/scholarly work in terms that
%% can be understood by the general public. It must not exceed 150
%% words in length.

\chapter{Lay Summary}

This nuclear physics research investigates the phenomenon of shape coexistence in atomic nuclei. The ways in which energetic nuclei get rid of excess energy, energy that makes their inhabited physical state an unstable one, can lead to an interpretation of their nuclear shape. The nuclei of different atoms are found in a range of spheroidal shapes, which can be characterized as different deformations of the simple, ball shape, from prolate (like a rugby ball), to oblate (like a lentil), and to yet more exotic configurations. Shape coexistence refers to the unique instance wherein a single nuclei can simultaneously coexist in multiple such shapes, a fundamentally quantum mechanical phenomenon. In this research the data from high-energy nuclear collisions, collected at TRIUMF, Canada's national particle accelerator centre, are analyzed. The conclusions as they pertain to this shape coexistence phenomenon are discussed. \newline

