%% The following is a directive for TeXShop to indicate the main file
%%!TEX root = diss.tex

\chapter{Abstract}

A measurement of the strength of the $E0$ transition between the $2_2^+$ and $2_1^+$ states of the nucleus $^{110}\mathrm{Pd}$ was performed. $E0$ transition strength measurements serve as a sensitive probe of the degree of nuclear shape mixing between different intrinsic structures within nuclei. Analysis was conducted of spectroscopic $\gamma$-ray and internal conversion electron data obtained by inelastic scattering of an accelerated alpha particle beam off of a $^{110}\mathrm{Pd}$ target. Internal conversion coefficients were measured for a range of transitions from the nuclei $^{110}\mathrm{Pd}$, $^{109}\mathrm{Pd}$, and $^{111}\mathrm{Ag}$. The $\gamma$-ray data was collected by the TRIUMF-ISAC Gamma-Ray Escape Spectrometer (TIGRESS), and the internal conversion electron data by the Spectrometer for Internal Conversion Electrons (SPICE). The use of these detectors in parallel required a study of the relative detection efficiencies, as a function of particle energy, such that absolute internal conversion coefficient measurements could be justified as accurate. In particular, SPICE was characterized further via a study of simulated particle detection efficiency using the Geant4 simulation toolkit. The resultant internal conversion coefficient measurements were combined with literature half life and branching ratio information to determine the $E0$ transition strength for the first time. A Monte Carlo error analysis approach was used to arrive at a measured limit on the true value of the $E0$ transition strength. This measured limit constitutes an effectively negligible $E0$ transition component, and evidence of little to no shape mixing between the intrinsic structures of the nucleus $^{110}\mathrm{Pd}$.

% Consider placing version information if you circulate multiple drafts
%\vfill
%\begin{center}
%\begin{sf}
%\fbox{Revision: \today}
%\end{sf}
%\end{center} 