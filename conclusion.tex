%% The following is a directive for TeXShop to indicate the main file
%%!TEX root = diss.tex




%%%%%%%%%%%%%%%%%%%%%%%%
\chapter{Conclusion}
\label{ch:Conclusion}
%%%%%%%%%%%%%%%%%%%%%%%%

In this work the $E0$ transition strength between the $2_2^+ \rightarrow 2_1^+$ transition in $^{110}$Pd was measured. This new value was obtained by measurement of the internal conversion coefficient for the $K$ atomic electron shell combined with literature values of the $(E2/M1)$ mixing ratio and parent state lifetimes. 
A setup using TIGRESS and SPICE at TRIUMF's ISAC-II facility was used to measure the electron-$\gamma$ branching ratios, or internal conversion coefficients of the aforementioned transition, in addition to several other previously unmeasured transitions from the nuclei $^{109}$Pd, $^{110}$Pd, and $^{111}$Ag. The measurement from the $2_2^+ \rightarrow 2_1^+$ transition in $^{110}$Pd was used as an input in calculating the $E0$ transition strength via Monte Carlo analysis. The methodology of performing the simulations and determining the confidence limits of the measurement was detailed at the end of \autoref{sec:Determining E0 Transition Strengths}. The $E0$ strength measured was effectively 0, and is characterized by the upper limit of $<8$. This limit gives further evidence to the absence of any shape coexistence within the nucleus $^{110}$Pd. This is consistent with expectations for transitions linking a $K=2$ band with the $K=0$ ground-state band.

Additionally, an upper limit has been given for the $\rho^2(E0)$ value of the $4^+_2 \rightarrow 4^+_1$ transition in $^{110}$Pd. The lack of a measured mixing ratio yields a relatively inexact upper bound value of $<129$, which represents an assumption of a maximal $\delta(E2/M1)$ value, i.e. negligible $M1$ component.

The inability to achieve a full measurement can be attributed to the large, asymmetric uncertainty of the $(E2/M1)$ mixing ratio. A more accurate measurement of this mixing ratio is possible using the data set from this experiment, and stands out as a definite direction for future work. 

\endinput

Any text after an \endinput is ignored.
You could put scraps here or things in progress.
